\documentclass[letterpaper,12pt]{article}
\usepackage[sfdefault]{cabin}
\usepackage[T1]{fontenc}
\usepackage{latexsym}
\usepackage[empty]{fullpage}
\usepackage{titlesec}
\usepackage{marvosym}
\usepackage[usenames,dvipsnames]{color}
\usepackage{verbatim}
\usepackage{enumitem}
\usepackage[hidelinks]{hyperref}
\usepackage{fancyhdr}
\usepackage[english]{babel}
\usepackage{tabularx}
\usepackage{fontawesome5}
\usepackage{multicol}
\usepackage{geometry}
\usepackage[pages=some]{background}
\setlength{\multicolsep}{-3.0pt}
\setlength{\columnsep}{-1pt}
\input{glyphtounicode}
\input{info.tex}

\backgroundsetup{
scale=1,
color=black,
opacity=1,
angle=0,
contents={%
  \includegraphics[width=\paperwidth,height=\paperheight]{figures/fondo.png}
  }%
}



\begin{document}

\BgThispage\vspace{15pt}

Estimado \firstName \lastName:\\ \vspace{15pt}



La Sociedad Científica Juvenil (SCJ), en colaboración con la Asociación Internacional de
Estudiantes de Física (IAPS), se complace en invitarte a participar como voluntario en la
trigésimo segunda edición del International Conference of Physics Students (ICPS),
evento que se celebrará del 9 al 16 de agosto del 2020 en Puebla de Zaragoza, México.
El ICPS es el evento magno de IAPS que se celebra anualmente desde 1985, cambiando
de sede cada año alrededor de Europa, siendo esta, la primera edición que se celebrará
en nuestro continente. El evento consiste en una serie de conferencias y talleres
relacionados con temas de física, además de actividades culturales que, a lo largo de
siete días, reúne a aproximadamente 500 estudiantes de más de 40 países.
Como miembros de SCJ y IAPS, nuestro objetivo es realizar eventos de calidad. El
comité organizador considera que tu destacada participación en las actividades de la
SCJ en tu sede, así como la motivación presentada en el formulario serán necesarias
para la realización del congreso. Como voluntario tendrás la responsabilidad de ayudar
con la logística para que el evento se lleve a cabo sin problemas. Las actividades y
horarios serán asignadas un par de días antes del evento.
Sin más por el momento, te agradecemos la atención prestada y esperamos contar con
su valiosa participación para hacer esta edición del ICPS, un evento que enaltezca la
ciencia en México.

\begin{center}
ATENTAMENTE \\ \vspace{10pt}
Comité organizador de ICPS Puebla 2020 \\ 
Sociedad Científica Juvenil \\
\end{center}



\end{document}
